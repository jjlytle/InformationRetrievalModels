\documentclass[sigconf]{acmart}
\graphicspath{ {images/} }
\usepackage{booktabs} % For formal tables



% Copyright
%\setcopyright{none}
%\setcopyright{acmcopyright}
%\setcopyright{acmlicensed}
\setcopyright{rightsretained}
%\setcopyright{usgov}
%\setcopyright{usgovmixed}
%\setcopyright{cagov}
%\setcopyright{cagovmixed}




% DOI
%\acmDOI{10.475/123_4}

% ISBN
%\acmISBN{123-4567-24-567/08/06}

%Conference
\acmConference[FRONTIERS'18]{ACM Computing Frontiers}{May 2018}{Ischia, Italy}
\acmYear{2018}
\copyrightyear{2018}


%\acmArticle{4}
%\acmPrice{15.00}

% These commands are optional
%\acmBooktitle{Transactions of the ACM Woodstock conference}
%\editor{Jennifer B. Sartor}
%\editor{Theo D'Hondt}
%\editor{Wolfgang De Meuter}


\begin{document}
\title{Indoor Localization within Hostile Environments}
%\title{Beacon Architecture Stuff}
%\titlenote{Produces the permission block, and
%  copyright information}
%\subtitlenote{The full version of the author's guide is available as
%  \texttt{acmart.pdf} document}


%\author{Ben Trovato}
%\authornote{Dr.~Trovato insisted his name be first.}
%\orcid{1234-5678-9012}
%\affiliation{%
%  \institution{Institute for Clarity in Documentation}
%  \streetaddress{P.O. Box 1212}
%  \city{Dublin}
%  \state{Ohio}
%  \postcode{43017-6221}
%}
%\email{trovato@corporation.com}


%\author{John Smith}
%\affiliation{\institution{The Th{\o}rv{\"a}ld Group}}
%\email{jsmith@affiliation.org}



% The default list of authors is too long for headers.
%\renewcommand{\shortauthors}{B. Trovato et al.}


\begin{abstract}




%\footnote{This is an abstract footnote}
\end{abstract}

%
% The code below should be generated by the tool at
% http://dl.acm.org/ccs.cfm
% Please copy and paste the code instead of the example below.
%
\begin{CCSXML}
<ccs2012>
 <concept>
  <concept_id>10010520.10010553.10010562</concept_id>
  <concept_desc>Computer systems organization~Embedded systems</concept_desc>
  <concept_significance>500</concept_significance>
 </concept>
 <concept>
  <concept_id>10010520.10010575.10010755</concept_id>
  <concept_desc>Computer systems organization~Redundancy</concept_desc>
  <concept_significance>300</concept_significance>
 </concept>
 <concept>
  <concept_id>10010520.10010553.10010554</concept_id>
  <concept_desc>Computer systems organization~Robotics</concept_desc>
  <concept_significance>100</concept_significance>
 </concept>
 <concept>
  <concept_id>10003033.10003083.10003095</concept_id>
  <concept_desc>Networks~Network reliability</concept_desc>
  <concept_significance>100</concept_significance>
 </concept>
</ccs2012>
\end{CCSXML}

\ccsdesc[500]{Computer systems organization~Embedded systems}
\ccsdesc[300]{Computer systems organization~Redundancy}
\ccsdesc{Computer systems organization~Robotics}
\ccsdesc[100]{Networks~Network reliability}

\keywords{Edge Computing, IoT, Systems}

\maketitle

\section{Introduction}

Matt - work this out. 

\section{Background}

Matt - add this. 

\subsection{Received Power Distance Estimation}

\subsection{Time of Flight}

\subsection{Angle of Arrival}

\subsection{Trilerateration}

\section{Related Work}

\section{Beacon Architecture \& Design}

\includegraphics[scale=0.25]{Anchor}

\includegraphics[scale=0.25]{Tag}

When designing hardware for the harsh conditions that first responders must face. There are many and varied considerations as well as restraints that must be tackled. First is the idea that any system were a first responders life may be at stake needs to be reliable. The responders will be required to carried these devices into harsh conditions so they must be light, small, and easy to deploy. Do to the conditions the first responders may face these modules may not be recoverable due to chemical contamination as well as deteriorating building conditions, so these devices must be somewhat disposable. The devices may be needed for a few minutes to many hours or possibly even days, so low power and long battery life is needed. An ideal device would be small, light weight, disposable, easy to deploy, low-cost, and most of all reliable. With these constraints we were able to conceptualize the hardware we needed to design.
One of the industry standards for developing reliability in hardware is know as (DFR) design for reliability. We have determined to confine our initial studies of reliability to this area, as when building a piece of hardware for first responders reliability takes on a life and death meaning. (DFR) or Design for Reliability is a set of tools that sets three important goals when designing hardware. First reliability must be designed into products and process using the best available science based methods. Two, Knowing how to calculate reliability is important, but knowing how to achieve reliability is equally, if not more, important. Three, Reliability practices must begin early in the design process and must be well integrated into the overall product development cycle.mThe tools we are initially focusing our efforts in are Voice Of the Customer (VOC), Design Of Experiments (DOE) , Failure Modes and Effects FMEA, Environmental Usage Conditions (EUC), and Accelerated Life Testing (ALT). As well as reliability demonstrations, collecting field data, failure analysis, using genuine parts from reputable suppliers and manufactures, and general brainstorming as this is a new technology and there is not predefined measures of reliability. Our general design cycle for maintaining reliability in the forefront of all our design decisions is below. The proposed system will use existing technologies such as tablets and on-site edge server to create a mobile dynamically placed indoor positioning system for first responders to coordinate safe efficient movement and rescues. The system will need to preform several functions such as environmental sensing, navigation, position, interfacing and sensing the vitals of first responders. 
The system consists of three main sections. The first being the dynamic positioning system for collecting position and vitals of first responders as wells environmental conditions, The second being an on-site edge server for collecting and interpreting the data streams from the position/sensing systems and finally a tablet for interfacing with the first responders.
The positioning portion of the system consists of anchor modules and tag modules. An anchor module is a sensor platform that also contains a data plane radio and a positioning radio. The system requires at least 3 of these anchors to be placed in known locations for the tag to be located by trilateration. The tag is also a sensor platform with the differnce being that is attached to the first responder and uses 3 or more anchors to locate itself and send it location out to the data planes mesh network. It also tracks and sends the vital signs of the first responders as well as the other sensor data. We are designing the anchors so that they can be placed dynamically by the first responders as they enter the building, but for now in the prototyping stage we are placing the anchors in known locations. The anchors are connected to a tag placed on the first responder by an Ultra wide band radio link. By using Symmetrical double-sided Two-Way Ranging (SDS-TWR) communication protocol between the tag and anchors 3 distances can be determined that can then be trilaterated by the on-board microprocessor to find the location of the receiving tag. The design of these modules had many necessary constraints placed on them the first of which always being reliability as this system is indented to be used in life critical situations so reliability was for most on the constraints list. These modules also needed to be light and portable to be brought into buildings by already over burdened first responders. They needed to be designed to be used in low or no visibility situations. A simple or no interface device would be appropriate. For our prototype design of this system we choose to first design and test an indoor positioning system. There are many designs for such a system such as wifi, bluetooth, RFID, 433 MHz and others. After testing these for positioning we finally deciding on the use of ultra wide band radio modules. The module we choose for our system was Decawaves DWM1000 module. It is an IEEE802.15.4-2011 UWB compliant wireless transceiver module based on Decawaves DW1000 IC with an integrated UWB antenna. The DWM1000 enables the tracking of objects in real time to a stated precision of 10 cm indoors. It has high data rate communications, up to 6.8 Mb/s, and an excellent communications range of up to 300 m thanks to coherent receiver techniques. This module is connected to a Atmel ATmega 328p over a standard I2C link. The ATmega was choose due to its large library of drivers for various sensors and parts including the DWM1000. In future iterations the MCU will be removed for a cheaper and more powerful option. The smaller MCU was chosen over larger more powerful single board computers due to the tight timing constraints needed to maintain the accuracy of positioning in the UWB radio as well as the speed at which we could get the system up and running. In a (VOC) voice of customer interview process with fire captain Mr.somebody, A number of different real world conditions were identified that if monitored would be beneficial to the safety and effciency of the crew. A number of different types of sensors were decided on. those that were recommended by talking to first responders were temperature, light, Co2, pulse and blood oxygen. The anchors will sense only light, Co2 and temperature. Where as the tags will have sensor for temperature, light, Co2, pulse and blood oxygen. The sensor choosen for temperature was the MCP9808T-E/MS which is from Microchips line of maximum accuracy digital temperature sensors. We choose this sensor because of its high reliability, long-term availability, programmable alert temperature, accuracy, low power requirerments , and non-colliding i2c address so it could be placed on the same i2c bus as the other sensors. For light sensing we choose the TSL2561-TCT-ND for its large sensing range from darkroom to high lux direct sunlight conditions. It is also a reliable, low-power, non-conflicting i2c device with long-term availability. For Co2 we choose the CCS811B ultra low power Volatile Organic Compound sensor. This sensor is a very compact total air quality sensor allowing the detection of alcohols, aldehydes, ketones, organic acids, amines, aliphatic and aromatic hydrocarbons. Which means it can detect irritants added to chemicals and smoke. It can also detect effective carbon dioxide from 400 to 8192 part per billion. The sensor we choose to detect both pulse and blood oxygen is the MAX30100. It is a low power digital pulse oximeter with a wide range for placement on the body including the forehead, so the sensor could be placed int the helmet of the first responder. All of these sensor are low cost as these devices will be entering harsh environmental conditions and may not be recoverable so design for disposability was another design constraint. 
    
    \includegraphics[scale=0.25]{GrowingReliability}

\subsection{RSSI-based Designs}

\subsection{UWB Design}

\begin{equation}
 distance=ToF*(speed of light)
\end{equation}

\begin{equation}
 ToF=[(TRR-TSP)-(TSR-TRP)+(TRF-TSR)-(TSF-TRR)]/4
\end{equation}

\includegraphics[scale=0.10]{Symmetrical_Double-Sided_Two-Way_Ranging}

Using Symmetrical double-sided Two-Way Ranging (SDS-TWR), Three messages must be passed between the two Ultra-wide band transceivers. The tag device worn by the first responder initiates the SDS-TWR by sending a Poll message to one of its currently registered Anchors. the message contains the TSP (Time of Sending Poll). Upon reception the receiving anchor records the (TRP) time of receiving poll and replies with a time-stamped messages of its own (TSR) time of sending response, including the TRP in it response. Upon receipt of the response the tag records the time of return response (TRR). The tag now creates the final message that contains the First responders' ID, TSP, TRR, TSF (time of sending final message) and sends this to the anchor. The anchor can now calculate the distance and then sends two messages the first being a message sent back to the tag so the tag can log its distance from the anchor. The anchors being a mesh network can forward this calculated distance. Which finds its way through the mesh network to the master anchor which passes the data to the on-site edge server. In future iterations of this protocol. We would like the final message of calculated distance sent back to the tag to also contain a new time-stamp so, that the tag and the anchor each have an independently calculated range that they can each compare. allowing for redundancy and error correction. This would also allow for a type of ranging protocol where the ping back and forth never stops and just the last term of range time is dropped. in this way both the tag and anchor always end up with the distance calculated in two ways. Tag to Anchor to Tag range as well as Anchor to Tag to Anchor range. With each successive message passed one or the other ranging TAT or ATA can be updated on either the tag or the anchor. This future update would allow redundancy of range measurement on both devices. The current iteration of having the Symmetrical double-sided Two-Way Ranging start at the tag instead of the anchor allows the anchor to end up with the most current range if the tag worn by the first responder where to lose contact with the anchor. In this way we can always have a last best guess as to where the first responder is.


\subsection{Components}

\subsection{Board Designs} 
\includegraphics[scale=0.10]{Schematic}

\subsection{Software Design}

\section{Evaluation Methodology}

Jeff, Hao, Bob - please work on describing the experimental methodology used in terms of the three configurations:  1) outside, 2) "open" building, and lab (worst case).  

\subsection{Operating Environments}

\subsection{Metrics}

\subsubsection{Accuracy}

\subsubsection{Static Positioning Drift}



\section{Evaluation Results}

Waiting on results from Hao and Bob for this section 

\subsection{Positioning Accuracy}

\subsection{Static Positioning Drift}




\section{Conclusions}

Future work. 




\begin{acks}
  This material is based upon work supported by the National Science Foundation under Grant No. CNS-\#1742899.  

\end{acks}


\bibliographystyle{ACM-Reference-Format}
\bibliography{sample-bibliography}


\end{document}
